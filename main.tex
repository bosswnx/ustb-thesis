\documentclass{ustb-thesis}

% \syntaxonly

\usepackage{graphicx}

% 论文基本信息
\newcommand{\ctitlefirst}{基于模糊测试的RISC-V}  % 中文标题第一行
\newcommand{\ctitlesecond}{SBI系统测试框架研究}  % 中文标题第二行
\newcommand{\etitlefirst}{A Research Based on XXXXX}  % 英文标题第一行
\newcommand{\etitlesecond}{XXXX System Research}  % 英文标题第二行
\newcommand{\collage}{计算机与通信工程学院}  % 学院
\newcommand{\major}{XXX专业}  % 专业
\newcommand{\class}{物联212}  % 班级
\newcommand{\name}{王诺贤}  % 作者
\newcommand{\id}{U202141934}  % 学号
\newcommand{\firstmentorname}{XXX}  % 第一指导老师姓名
\newcommand{\firstmentortitle}{教授}  % 第一指导老师职称
\newcommand{\secondmentorname}{XXX}  % 第二指导老师姓名
\newcommand{\secondmentortitle}{副教授}  % 第二指导老师职称
\renewcommand{\year}{2025}  % 年份
\renewcommand{\month}{05}  % 月份
\newcommand{\miji}{公开}  % 密级
\newcommand{\ckeywords}{模糊测试, RISC-V, SBI}  % 中文关键词
\newcommand{\ekeywords}{Fuzzing, RISC-V, SBI}  % 英文关键词


\begin{document}
\pagenumbering{gobble}
\input{src/1-cover.tex}

\cleardoublepage

\input{src/2-title.tex}

\cleardoublepage

% 声明
\includepdf[pages=-]{src/3-statement.pdf}

\cleardoublepage

% 任务书
\includepdf[pages=-]{src/4-taskplan.pdf}

\cleardoublepage

\pagenumbering{Roman} % 罗马页码
\input{src/5-abstract.tex}

\cleardoublepage

\tableofcontents % 生成目录

\cleardoublepage

\input{content/figures-and-tables.tex}  % 插图或附表清单页面,不需要时请删除此行

\cleardoublepage

\input{content/notes-and-explanations.tex}  % 注释说明清单页面,不需要时请删除此行

\cleardoublepage

\pagenumbering{arabic} % 重置页码

% 正文页面,删除添加章节请修改这里的内容
\input{content/ch1.tex}
\input{content/ch2.tex}
\input{content/ch3.tex}
\input{content/ch4.tex}
\input{content/ch5.tex}

\cleardoublepage

% 参考文献
\addcontentsline{toc}{chapter}{参考文献}
\printbibliography

\cleardoublepage

% 附录
\input{content/appendices.tex}

\cleardoublepage

% 在学取得成果
\input{content/achievements.tex}

\cleardoublepage

% 致谢
\input{content/thanks.tex}

\clearpage

\end{document}

