% 附录
\addcontentsline{toc}{chapter}{附录A}  % 手动添加目录
\chapter*{附录A}

根据需要,可以再增加附录,比如“附录A”、 “附录B”、“附录C”等,比如文中的原始数据、文中的调查问卷样板、文中的程序清单、文中的公式推导等。

本部分在格式上应与本毕业论文的正文相协调,应保持基本美观。文字基本采用“b正文”样式、“b图标题” 样式、“b表标题” 样式等,1级标题建议采用“黑体、小三”,2 级、3级标题建议采用“黑体、四号”,图、表、公式等应清晰、字体不易过大过小,行距基本为1.3倍行距。 

1 级标题采用“A.1”、 “A.2”、“A.3”等编号方式,2级标题采用“A.1.1”、 “A.1.2”、“A.2.1”等编号方式,3级标题的编号方式类似。图表标题的编号类似,如“图A.1”、“表A.3”等。公式编号类似,如“(式A.1)”。

% 注意:由于附录的具体章节不编入目录,因此请使用 \addcontentsline 命令将附录的一级标题手动添加到目录中。