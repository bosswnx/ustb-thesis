\chapter{结\hspace{1em}论}

论文应有结论。论文的结论是最终的、总体的结论,不是正文中各段的小结的简单重复。

结论应包括论文的核心观点,列出论文的创新之处,交待研究工作的局限,提出未来研究工作的意见或建议。

结论应该观点明确、严谨、完整、准确、精炼。文字必须简明扼要。如果不可能导出应有的结论,也可以没有结论而进行必要的讨论。

结论是论文的“收尾之笔”,应是“点睛之笔”,应认真阐明本人在科研工作中创造性的成果和新见解,在本领域中的地位和作用,新见解的意义。结论中不要简单重复罗列实验结果,要对存在的问题和不足作出客观的叙述,并提出进一步的设想。应严格区分自己的成果与他人(特别是导师的)科研成果的界限。